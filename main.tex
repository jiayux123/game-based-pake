\documentclass{article}

\usepackage{amsmath}
\usepackage{amsfonts}
\usepackage{amsthm}
\newtheorem{definition}{Definition}
\newtheorem{theorem}{Theorem}
\newtheorem{lemma}{Lemma}
\newtheorem{claim}{Claim}
\usepackage{authblk}
\usepackage{cryptocode}
\usepackage{csquotes}
\usepackage{enumerate}
\usepackage{float}
\usepackage{framed}
\usepackage[colorlinks]{hyperref}
\def\tmp#1#2#3{
  \definecolor{Hy#1color}{#2}{#3}
  \hypersetup{#1color=Hy#1color}}
\tmp{link}{HTML}{800006}
\tmp{cite}{HTML}{2E7E2A} % better default colors for hyperref links; see https://tex.stackexchange.com/a/525297
\usepackage[capitalise]{cleveref}
\usepackage{ocgx} % spoiler hack
\usepackage{tikz-cd}

\newcommand{\group}{\mathbb{G}}
\newcommand{\Z}{\mathbb{Z}}

\newcommand{\negl}{\mathsf{negl}}

\newcommand{\adv}{\mathcal{A}}
\newcommand{\env}{\mathcal{Z}}
\renewcommand{\sim}{\mathcal{S}}
\newcommand{\red}{\mathcal{R}}
\newcommand{\func}{\mathcal{F}}
\newcommand{\Fpake}{\func_\mathrm{PAKE}}
\newcommand{\Fipake}{\func_\mathrm{iPAKE}}

\newcommand{\Adv}{\mathbf{Adv}}
\newcommand{\Dist}{\mathbf{Dist}}

\newcommand{\ICEnc}{\mathcal{E}}
\newcommand{\ICDec}{\mathcal{D}}

\newcommand{\pw}{\mathsf{pw}}

\newcommand{\NewSession}{\mathsf{NewSession}}
\newcommand{\TestPwd}{\mathsf{TestPwd}}
\newcommand{\NewKey}{\mathsf{NewKey}}

\newcommand{\fresh}{\mathsf{fresh}}
\newcommand{\compromised}{\mathsf{compromised}}
\newcommand{\interrupted}{\mathsf{interrupted}}
\newcommand{\completed}{\mathsf{completed}}

\newcommand{\Execute}{\mathsf{Execute}}
\newcommand{\Send}{\mathsf{Send}}
\newcommand{\Reveal}{\mathsf{Reveal}}
\newcommand{\Test}{\mathsf{Test}}

\newcommand{\SpuriousGuess}{\mathsf{SpuriousGuess}}
\newcommand{\CorrectGuess}{\mathsf{CorrectGuess}}

\interfootnotelinepenalty=10000 % force footnotes to appear on the same page

\begin{document}

\title{Does the UC-Security Notion for PAKE \\ Imply Game-Based Security?
\thanks{The \texttt{.tex} code of this paper can be found at \url{https://github.com/jiayux123/game-based-pake}.}
}

\author{Jiayu Xu}
\affil{Oregon State University, \texttt{xujiay@oregonstate.edu}}
\date{}

\maketitle

\begin{abstract}
A Password-Authenticated Key Exchange (PAKE) protocol allows two parties to jointly establish a cryptographically strong key, in the setting where the only information shared in advance is a low-entropy ``password''. The two standard security definitions for PAKE are the game-based one by Bellare, Pointcheval and Rogaway (BPR-security, EUROCRYPT 2000) and the Universally Composable (UC) one by Canetti et al. (EUROCRYPT 2005). It is well-known that UC-security implies BPR-security; however, there are a large number of variants of both definitions, and the relation between them is not entirely clear.

In this work, we thoroughly study a variant of BPR-security by Katz, Ostrovsky and Yung (KOY-security, JACM 2009):
\begin{enumerate}
  \item We show, via a counterexample, that UC-security does \emph{not} imply KOY-security;
  \item We then prove that a variant of UC-security, called implicit-only UC-security (Dupont et al., EUROCRYPT 2018), implies KOY-security.
\end{enumerate}
Interestingly, we make the observation that KOY- and implicit-only UC-security essentially strengthen their standard counterparts in the same manner. We also present detailed explanations of all four security notions.
\end{abstract}
\section{Introduction}
A \emph{Password-Authenticated Key Exchange (PAKE)} protocol allows two parties to jointly establish a cryptographically strong key, in the setting where the only information shared in advance is a low-entropy string called the password. Crucially, such a protocol must be secure against man-in-the-middle adversaries that can arbitrarily modify protocol messages. One common attack on PAKE is \emph{online guessing}, in which the adversary chooses a candidate password $\pw^*$ and impersonates Bob to Alice; if Alice's password is indeed $\pw^*$, then the adversary learns Alice's output key. Since passwords have low entropy, the adversary has a non-negligible probability of guessing Alice's password correctly; at a high level, security for PAKE requires that online guessing is the only feasible attack.

While the intuition of PAKE security looks simple and innocuous, a formal security definition proves incredibly difficult. There are two such definitions that are considered standard: the game-based security notion by Bellare, Pointcheval and Rogaway \cite{EC:BelPoiRog00} (henceforth \emph{BPR-security}), and the Universally Composable (UC) security notion by Canetti et al. \cite{EC:CHKLM05} (henceforth \emph{UC-security}). Both definitions take pages to describe and explain, and contain various subtleties that are confusing to beginners and understudied by the community. A standard result \cite[Appendix~A]{EC:CHKLM05} shows that UC-security implies BPR-security.

Over the years, a large number of variants of both BPR- and UC-security have been proposed. While many of these are essentially tailoring the definition to fit a specific protocol construction, some do have an intuitive interpretation and capture certain nuances of PAKE security. This work focuses on two of them:
\begin{itemize}
  \item The variant of BPR-security by Katz, Ostrovsky and Yung \cite{JACM:KatOstYun09} (henceforth \emph{KOY-security}), which places more constraints on what exactly counts as an ``online attack''; and
  \item The variant of UC-security by Dupont et al. \cite{EC:DHPRY18} called \emph{implicit-only UC-security}, which disallows the ideal adversary from learning whether an online attack has succeeded or not.
\end{itemize}
Neither of the aforementioned works' main focus was on security notions. The conference proceedings version of the KOY paper \cite{EC:KatOstYun01} proposed a new PAKE protocol and proved its BPR-security; in the journal version \cite{JACM:KatOstYun09} KOY proved that their protocol actually satisfies the stronger KOY-security, although the difference between these two notions was left unexplained. Implicit-only UC-security was introduced because the authors of \cite{EC:DHPRY18} used PAKE to construct what's called fuzzy PAKE, and the underlying PAKE needs to satisfy this stronger security notion for their fuzzy PAKE protocols to be secure. Perhaps this explains why these two security notions are not widely used: we do not know of any studies of either KOY- or implicit-only UC-security per se, and subsequent works generally continued to use BPR- and UC-security.

\paragraph{Our contributions.}
In this work, we present a thorough study of KOY- and implicit-only UC-security for PAKE, and draw some interesting connections between them and their more standard counterparts. Our technical contributions are two-fold:
\begin{enumerate}
  \item In \Cref{sec:counterexample} we show that UC-security does \emph{not} imply KOY-security, by giving a concrete counterexample that is UC-secure but not KOY-secure;
  \item In \Cref{sec:implicit-only} we show that implicit-only UC-security implies KOY-security.
\end{enumerate}
\usetikzlibrary{decorations.markings}
\tikzset{negated/.style={
        decoration={markings,
            mark= at position 0.5 with {
                \node[transform shape] (tempnode) {$\backslash$};
            }
        },
        postaction={decorate}
    }
}
\begin{figure}[H]
\begin{center}
\begin{tikzcd}[row sep = 8ex, column sep = 8ex]
\text{implicit-only UC} \arrow{r} \arrow{d}[swap]{\text{\Cref{sec:implicit-only}}} & \text{UC} \arrow{d}{\text{\cite{EC:CHKLM05}}} \arrow[negated]{dl}{\text{\Cref{sec:counterexample}}} \\
\text{KOY} \arrow{r} & \text{BPR}
\end{tikzcd}
\end{center}
\end{figure}
Finally, in \Cref{sec:conclusion} we point out that KOY- and implicit-only UC-security essentially strengthen their standard counterparts in the same manner, namely \switchocg{name_int}{[click to reveal spoiler]}
\begin{ocg}{name_ext}{name_int}{0}\emph{disallowing the adversary from learning whether an online attack has succeeded or not by merely examining the protocol transcript (without seeing the attacked party's output key)}\end{ocg}. In this way, we give an interpretation of what exactly the KOY- and implicit-only UC-security notions entail.

As a side contribution, in \Cref{sec:prelims} we discuss the four security definitions in detail --- not only \emph{what} exactly these definitions are, but also the \emph{rationales} behind certain design choices in the definitions. By doing this, we hope to help ``demystify'' PAKE security notions and make them more friendly to beginners; as such, this work also serves as a survey (or tutorial) of different PAKE security definitions --- which is incomplete but does cover the standard ones that are most commonly used.

\paragraph{A personal anecdote: how did I dig out KOY-security?}
I learned game-based PAKE security definitions ``backwards''. For a very long time, KOY-security was the standard reference for me, and I had no idea that KOY and BPR were actually different. Very recently, I needed to prove a generalization of UC-security $\Rightarrow$ game-based security, and I just couldn't make the proof go through. After some thought, I realized that the ``theorem'' I tried to prove was actually wrong --- similar to the counterexample in \Cref{sec:counterexample} --- and you needed to tweak the game-based security definition to make the proof work.

At that time I thought there was an error in the standard result in \cite{EC:CHKLM05}. \cite{EC:CHKLM05} claims that the game-based security notion they use is BPR, so I checked \cite{EC:BelPoiRog00}. I immediately realized that the ``BPR-security'' I had in mind was in fact not what BPR said, and the actual BPR-security was exactly what I needed in my proof.

I decided to write down my counterexample. During this process, I realized that apart from replacing KOY- with BPR-security, another way to invalidate the counterexample would be to strengthen UC-security to implicit-only UC-security. Perhaps implicit-only UC-security implies KOY-security? This proof indeed went through, and the technical pieces of this work were complete.

\section{Preliminaries}
\label{sec:prelims}
Let $\lambda$ be the security parameter. Most notations in this work are standard: for example, $x \gets S$ means sampling an element $x$ uniformly at random from set $S$; $x := a$ means assigning value $a$ to variable $x$; PPT means ``probabilistic polynomial-time''; and $\negl$ denotes a negligible function in $\lambda$. We assume the reader is familiar with the UC framework \cite{FOCS:Canetti01}.
\subsection{Overview of Security Definitions for PAKE}
\label{sec:principles}
We first give an informal, high-level overview of the ideas behind the security definitions for PAKE; what's said in this section applies to both the game-based and UC definitions. A few paragraphs in this section and \Cref{sec:UC} are taken almost verbatim from \cite{EPRINT:Xu25}.

Any reasonable security definition for PAKE must formalize the following two principles:
\begin{framed}
\begin{displayquote}
  \emph{The first principle: The adversary can only perform a single online guessing attack per protocol instance.}
  
  \emph{The second principle: All output keys are independent of each other, except in cases where correlation is inevitable.}
\end{displayquote}
\end{framed}
Correspondingly, the definition must contain the following parts (which is where the technical complications come from):
\begin{enumerate}
  \item What exactly constitutes an ``online guessing attack'' must be defined.
  \item Each protocol instance should output a uniformly random key independent of everything else, except that the cases where correlation is inevitable need to be singled out and handled separately. 
\end{enumerate}

The second principle warrants further explanation. What are those cases where correlation of output keys is inevitable? There are two of them:
\begin{itemize}
  \item If the adversary succeeds in an online attack, then of course it can learn the attacked instance's output key. In this case all security guarantees for this instance are lost and we cannot claim independence;
  \item If the adversary is passive between two instances (i.e., it merely forwards all protocol messages without any modification), then of course these two instances output the same key. (But this key should still be independent of other instances' output keys.)\footnote{This assumes that the two instances' passwords match. As we will see, in the game-based definitions we always assume this, whereas in the UC definitions we additionally consider the case where the two instances (that intend to connect to each other) hold different passwords.}
\end{itemize}

We will see that the above forms the core of both the game-based and UC definitions; however, they formalize these two principles in drastically different ways.
\subsection{BPR- And KOY-Security Definitions for PAKE}
\paragraph{Setup.}
The system consists of a number of protocol parties $P$, and each party can have a number of instances $P^i$. An instance can send protocol messages to an instance that belongs to another party.

Let $D$ be the set of all candidate passwords (the ``dictionary''); we assume $|D| \geq 2$ and is polynomial, and $D$ is public. For each pair of instances $(P^i, (P')^j)$, sample $\pw_{i,j} = \pw_{j,i} \gets D$ as the shared password between them.\footnote{We write $P^i$ and $(P')^j$ to stress that if two instances want to communicate, they must belong to different parties. Both \cite{EC:BelPoiRog00} and \cite{JACM:KatOstYun09} partition all parties into two categories, clients and servers, and a client instance must communicate with a server instance (and vice versa). Our treatment is slightly more general (e.g., it allows Alice to be a client while talking to Bob and a server while talking to Carol).} (Note that we do not consider the case where the two instances hold different passwords.)

\paragraph{Protocol execution.}
The security notion for PAKE is formalized as a game between a PPT adversary $\adv$ and a game challenger, where the challenger runs the protocol parties (and their instances) and sends to $\adv$ certain information that $\adv$ is supposed to learn during an attack. Concretely, $\adv$ can send the following commands to the game challenger:
\begin{itemize}
  \item $\Execute(P^i, (P')^j)$ --- This models an eavesdropping attack, where the adversary is passive and learns all protocol messages sent between the two instances (the ``protocol transcript''). Concretely, if neither $P^i$ nor $(P')^j$ is used, the game challenger runs the protocol between $P^i$ and $(P')^j$, and returns the transcript to $\adv$. Once a $\Execute(P^i, (P')^j)$ command is sent, the game challenger marks both $P^i$ and $(P')^j$ as used.
  \item $\Send(P^i, (P')^j, m^*)$ --- This models a man-in-the-middle attack, where the adversary sends message $m^*$ to $(P')^j$, as if the message is from $P^i$ (below we simply say ``the adversary sends message $m^*$ from $P^i$ to $(P')^j$''). Concretely, the game challenger sends message $m^*$ from $P^i$ to $(P')^j$; if $(P')^j$ sends a protocol message $m'$ then the game challenger returns $m'$ to $\adv$, otherwise (i.e., $m^*$ is the last message in the protocol causing $(P')^j$ to terminate) the game challenger returns $\bot$ to $\adv$. \\
      As a special case, $\Send(P^i, (P')^j, \text{start})$ models the adversary initiating instance $P^i$ and sees its first message (intended for $(P')^j$). Concretely, if $P^i$ is not used, the game challenger initiates instance $P^i$ and returns $P^i$'s message to $\adv$. Once a $\Send(P^i, (P')^j, \text{start})$ command is sent, the game challenger marks $P^i$ as used.\footnote{Our specification assumes that if $(P')^j$ is the responder, i.e., $(P')^j$ waits for $P^i$'s message before sending its own message, then the first of $\Send((P')^j, P^i, \text{start})$ and $\Send(P^i, (P')^j, m^*)$ returns $\bot$ and the second returns $(P')^j$'s message to $P^i$. (In other words, in order to see the responder's message, $\adv$ has to send both a ``start'' command for the responder and a message from the initiator to the responder.) This slightly deviates from \cite{EC:BelPoiRog00,JACM:KatOstYun09} but fits the UC-security definition better.}
\end{itemize}
Note that how we handle a used instance (i.e., an instance that has been initiated) guarantees that an instance cannot be run twice; in particular, $\adv$ cannot send both $\Execute$ and $\Send$ on the same instance. Therefore, all used instances can be partitioned into two types: $\Execute$ instances and $\Send$ instances.
\begin{itemize}
  \item $\Reveal(P^i)$ --- This models the adversary learning a completed instance's output key, via e.g., a side-channel attack. Concretely, if $P^i$ has computed its output key $SK_i$, the game challenger returns $SK_i$ to $\adv$; otherwise the game challenger returns $\bot$ to $\adv$.
\end{itemize}

This completes the description of what $\adv$ can see while attacking the protocol.
\paragraph{Security.}
We still need to decide whether $\adv$ wins the game. Similar to the security definition for unauthenticated key exchange, we let $\adv$ choose an instance and try to distinguish its output key with a random string:
\begin{itemize}
  \item $\Test(P^i)$ --- The game challenger samples a bit $b \gets \{0,1\}$. If $P^i$ has computed its output key $SK_i$ and $b = 1$, the game challenger returns $SK_i$ to $\adv$; if $P^i$ has computed its output key $SK_i$ and $b = 0$, the game challenger returns a uniformly random string in $\{0,1\}^\lambda$ to $\adv$; otherwise the game challenger returns $\bot$ to $\adv$. The $\Test$ command can be sent only once.
\end{itemize}
At the end of the game, $\adv$ outputs a bit $b^*$ and wins if $b^* = b$.

If we stop here then there are a few ways for $\adv$ to win trivially, which we must disallow:
\begin{enumerate}
  \item If $\adv$ sends $\Reveal(P^i)$ and $\Test(P^i)$ for the same $P^i$ (no matter what the order is), then $\adv$ knows that $b = 1$ if and only if they return the same result (except with negligible probability --- same below).
  \item If $\adv$ sends $\Execute(P^i, (P')^j)$, then $P^i$ and $(P')^j$'s output keys are the same. Then if $\adv$ $\Reveal$s one of them and $\Test$s the other, then again $b = 1$ if and only if they return the same result.
  \item Note that $\Reveal$ is not the only way for $\adv$ to be passive: if both $P^i$ and $(P')^j$ are $\Send$ instances, and their views are identical (i.e., $\adv$ forwards all messages between them via $\Send$ commands), then $P^i$ and $(P')^j$'s output keys are again the same. If $\adv$ $\Reveal$s one of them and $\Test$s the other, then $b = 1$ if and only if they return the same result.\footnote{This does not consider the case where $\adv$ is ``effectively passive'' by merely modifying the \emph{format} of protocol messages. For example, if a protocol message contains a bit $\mathsf{bit}$ that is ignored by the receiver, then $\adv$ can flip $\mathsf{bit}$ while transmitting this message; this should be treated as if $\adv$ is passive. We do not consider such degenerate cases, as they are inconsequential to our main results; for how to talk about such cases formally, see the discussion about ``session ID'' (not to be confused with the session ID in the UC framework) in \cite{EC:BelPoiRog00,JACM:KatOstYun09}.}
\end{enumerate}
We need to prohibit all these behaviors, i.e., we only quantify over all PPT adversaries that do not do any of 1--3 above.\footnote{\cite{EC:BelPoiRog00,JACM:KatOstYun09} define two instances to be \emph{partnered} if their transcripts are the same. Using this term, 1--3 above can be succinctly stated as: For the instance that $\adv$ $\Test$s, $\adv$ cannot $\Reveal$ either the instance itself or its partner, no matter whether $\Reveal$ happens first or $\Test$ happens first.}

Note that 2 and 3 above exactly reflect the second principle in \Cref{sec:principles}: if $\adv$ is passive between $P^i$ and $(P')^j$, then their output keys will be the same, and we need to place some restrictions on what $\adv$ can $\Test$. PAKE security should mean that except for these cases, the instances' output keys are uniformly random and independent of everything else, so $\adv$ can $\Test$ any one of them and the real key should be indistinguishable from a random string --- unless $\adv$ succeeds in an online attack, which we define next.
\paragraph{The ``baseline'' winning probability.}
The last piece of the definition is to define $\adv$'s ``baseline'' winning probability, i.e., $\adv$'s winning probability if it only performs inevitable attacks. The first principle in \Cref{sec:principles} states that each instance should allow only one online attack, in which $\adv$ can check one password guess (and gain $1/|D|$ advantage); therefore, if there are $q$ instances that are online attacked, then the ``baseline'' winning probability should be $1/2 + q/2|D|$.

But when exactly is an instance ``online attacked''? This is where BPR- and KOY-security diverge. To begin with, if the instance is not even used (i.e., initiated) then of course there is no way to attack it. For an $\Execute$ instance, since $\adv$ is passive there and merely sees the transcript, that should not count as an online attack either.\footnote{This implies an important security property for PAKE: a PAKE protocol should be resilient to ``offline dictionary attacks'', i.e., a protocol transcript that the adversary passively observes should not tell the adversary any information about the password. For example, $H(\pw)$ (where $H$ is an RO) cannot be used as a protocol message.}

This leaves $\Send$ instances. {\color{red}In BPR-security, all $\Send$ instances are considered online attacked}; {\color{blue}in KOY-security, a $\Send$ instance is online attacked only if $\adv$ also $\Reveal$s or $\Test$s it}. (The reader should pause a bit here and think about what this difference actually means; it should become clear after seeing the counterexample in \Cref{sec:counterexample}, and we reveal our understanding in \Cref{sec:conclusion}.) With the concept of ``online attacks'' defined, we can finally state our security definition for PAKE.
\begin{definition}
A PAKE protocol is \emph{\textsf{BPR-secure/KOY-secure}} if for any PPT adversary $\adv$ that does not perform any of the prohibited behaviors 1--3 above,
\[
\Pr[\adv\text{ wins}] \leq \frac{1}{2} + \frac{q}{2|D|} + \negl,
\]
where $q$ is the upper bound of the number of instances that are online attacked. The definition of an ``online attacked instance'' is the red text above for BPR-security and the blue text above for KOY-security.
\end{definition}
We stress that the security game for BPR- and KOY-security is exactly the same; the only difference lies in what counts as an ``online attack'', which in turn affects the ``baseline'' winning probability. KOY-security is stronger because it places more restrictions on ``online attacks'', causing $q$ --- hence the ``baseline'' winning probability --- to be potentially lower (so it is easier for $\adv$ to ``break'' the protocol).
\subsection{UC- And Implicit-Only UC-Security Definitions for PAKE}
\label{sec:UC}
UC-security definitions are done via a functionality that defines the ideal adversary's (a.k.a. the simulator's) interface. See the UC- and implicit-only UC functionalities for PAKE in \Cref{fig:pake-func}.
\begin{figure}[H]
\begin{framed}
\begin{small}
\begin{itemize}
  \item On input $(\NewSession, sid, P, P', \pw)$ from $P$, send $(\NewSession, sid, P, P')$ to $\adv^*$. Furthermore, if this is the first $\NewSession$ message for $sid$, or this is the second $\NewSession$ message for $sid$ and there is a record $\left<P',P,\star\right>$, then store $\left<P,P',\pw\right>$ and mark it $\fresh$.
  \item On $(\TestPwd, sid, P, \pw^*)$ from $\adv^*$, if there is a record $\left<P,P',\pw\right>$ marked $\fresh$, then do: \quad \textcolor[gray]{0.5}{// $\pw$ is $P$'s password}
      \begin{itemize}
        \item If $\pw^* = \pw$, then mark the record $\compromised$ {\color{red}and send ``correct guess'' to $\adv^*$}.
        \item If $\pw^* \neq \pw$, then mark the record $\interrupted$ {\color{red}and send ``wrong guess'' to $\adv^*$}.
      \end{itemize}
        \item On $(\NewKey, sid, P, SK^* \in \{0,1\}^\lambda)$ from $\adv^*$, if there is a record $\left<P,P',\pw\right>$, and this is the first $\NewKey$ message for $sid$ and $P$, then output $(sid, SK)$ to $P$, where $SK$ is defined as follows:
            \begin{itemize}
              \item If the record is $\compromised$, then set $SK := SK^*$.
              \item If the record is $\fresh$, a key $(sid, SK')$ has been output to $P'$, at which time there was a record $\left<P',P,\pw\right>$ marked $\fresh$, then set $SK := SK'$. \quad \textcolor[gray]{0.5}{// if no attack in this session, and passwords match, then keys should also match}
              \item Otherwise sample $SK \gets \{0,1\}^\lambda$.
            \end{itemize}
            Finally, mark the record $\completed$. \quad \textcolor[gray]{0.5}{// cannot send $\TestPwd$ after instance completes}
    \end{itemize}
\end{small}
\end{framed}
\caption{UC PAKE functionality $\Fpake$ (with red text) and implicit-only UC PAKE functionality $\Fipake$ (red text omitted)}
\label{fig:pake-func}
\end{figure}

In the UC functionalities, party $P$'s instance is represented by the record $\left<P,P',\pw\right>$, where $P'$ is $P$'s counterparty and $\pw$ is $P$'s password. A $\NewSession$ command initiates an instance; each session has two instances, so we allow for a $\NewSession$ command that initiates the $\left<P,P',\pw\right>$ instance and another $\NewSession$ command that initiates the $\left<P',P,\pw'\right>$ instance. Note that $P$ and $P'$'s passwords might be different.

Online guessing attacks are modeled by the $\TestPwd$ command, in which the ideal adversary specifies an instance and a password guess. Crucially, a $\TestPwd$ command can be sent only when the instance is $\fresh$, and once such a command is sent, the instance will become $\compromised$ or $\interrupted$ (depending on whether the password guess is correct) and never return to $\fresh$. (Think of $\fresh$ as a shorthand for ``unattacked'', $\compromised$ as ``successfully attacked'', and $\interrupted$ as ``unsuccessfully attacked''.) In this way, it is guaranteed that at most one $\TestPwd$ command can be sent per instance: this is how the UC functionality reflects the first principle in \Cref{sec:principles}.

$\TestPwd$ is also where the UC- and implicit-only UC-security definitions differ: in UC the ideal adversary knows whether its password guess is correct or not, whereas in implicit-only UC it doesn't. We will discuss what this entails in \Cref{sec:conclusion}. Clearly, implicit-only UC-security is stronger than UC-security, as the simulator has less power in the former (making the simulation potentially harder).

Finally, a $\NewKey$ command allows an instance to complete and output its key. How this output key is generated exactly reflects the second principle in \Cref{sec:principles}:
\begin{itemize}
  \item If the instance is $\compromised$, i.e., the adversary has succeeded in an online attack, then all security guarantees for this instance are lost. In this case, we simply let the ideal adversary set the output key (the first bullet).
  \item If both the current instance and its counterparty's instance are $\fresh$, i.e., the adversary is passive between two instances, and their passwords match, then these two instances should output the same random key. Formalizing this is a little cumbersome: if $P'$ outputs before $P$ does, then $P'$ should output a random key independent of everything else (a subcase of the third bullet), and when $P$ outputs later, its key should be equal to $P'$'s key (the second bullet).
  \item In all other cases, the instance's output key should be independent of everything else (the third bullet).
\end{itemize}
\begin{figure}[H]
\begin{center}
$
\begin{array}{c c c c c c c}
\text{command} & & \NewSession & \longrightarrow & \TestPwd & \longrightarrow & \NewKey \\
\\
\text{state} & & \fresh & \longrightarrow & \compromised & \longrightarrow & \completed \\
 & & & & \text{or }\interrupted & &
\end{array}
$
\end{center}
\caption{``Life cycle'' of an instance. $\TestPwd$ can be skipped}
\end{figure}
\paragraph{Comparison with game-based definitions.}
Both game-based and UC-security definitions formalize the two principles in \Cref{sec:principles}. However, they do so in very different ways.

The game-based definitions reflect the first principle in the ``baseline'' winning probability: an online attack should only increase the adversary's advantage by up to $1/|D|$. In contrast, the UC definition directly lets the ideal adversary specify a password guess while doing an online attack, and the ideal adversary can send at most one $\TestPwd$ command per instance (subsequent such commands will be ignored).

For the second principle, the technical challenge is to rule out the special cases where output key correlation is inevitable. The game-based definitions prohibit the adversary from $\Reveal$ing one instance and $\Test$ing its counterparty's instance, if the adversary is passive between the two instances; as for successfully attacked instances, the game simply ``gives up'' and allows the adversary to win (because this possibility is already covered by the ``baseline'' winning probability). The UC definitions again reflect this principle explicitly: the ``successfully attacked instances'' are handled in the first bullet under $\NewKey$, and the ``passive adversary'' is handled in the second bullet.

Finally, the game-based definitions always assume two instances that connect to each other have the same password, whereas the UC definitions allow the two parties to have different passwords. In particular, what will happen if the adversary is passive between two instances with different passwords? The game-based definitions do not consider this situation, whereas the UC definitions specify that the two instances' output keys must be independently random in this case.
\subsection{The EKE Protocol}
The \emph{Encrypted Key Exchange (EKE) protocol} \cite{SP:BelMer92} is the first PAKE protocol ever proposed. Here we will use EKE based on Diffie--Hellman, in which party $P$ samples $x \gets \Z_p$ and sends $c := \ICEnc(\pw, g^x)$ to its counterparty $P'$, and on $c'$ from $P'$, $P$ computes $Y := \ICDec(\pw, c')$ and outputs $\text{Hash}(Y^x)$ as the key; $P'$ does the same. Here, $(\group, g, p)$ is a cyclic group with generator $g$ and order $p$, $(\ICEnc, \ICDec)$ is an ideal cipher, and $\text{Hash}$ is an RO; both have range $\{0,1\}^\lambda$.
\begin{figure}[H]
\pseudocodeblock{
P \< \< P' \\ 
x \gets \Z_p; X := g^x \< \< y \gets \Z_p; Y := g^y \\
\< \sendmessageright*{c := \ICEnc(\pw,X)} \< \\
\< \sendmessageleft*{c' := \ICEnc(\pw,Y)} \< \\
Y := \ICDec(\pw,c') \< \< X := \ICDec(\pw,c) \\
K := \text{Hash}(Y^x) \< \< K' := \text{Hash}(X^y) \\
\text{output }K \< \< \text{output }K'
}
\caption{The EKE protocol}
\end{figure}

EKE has been proven to UC-realize the (standard) UC PAKE functionality $\Fpake$ under the CDH assumption in $(\group, g, p)$ \cite{EC:JanRoyXu25}. We specify the EKE protocol for concreteness only; it is only used in our counterexample in \Cref{sec:counterexample}, and can be replaced by any ``normal'' PAKE protocol that UC-realizes $\Fpake$.
\section{UC-Security Does Not Imply KOY-Security}
\label{sec:counterexample}
\subsection{The Counterexample}
Consider the following PAKE protocol:

\bigskip\noindent\textbf{\underline{Protocol $\Pi$}:}
\begin{enumerate}
  \item Protocol parties $P$ and $P'$ run EKE, resulting in keys $K$ (for $P$) and $K'$ (for $P'$).
  \item $P$ sends $h := H_0(K)$ to $P'$.
  \item $P$ and $P'$ outputs $H(K)$ and $H(K')$, respectively. Here, $H$ and $H_0$ are two ROs from $\{0,1\}^\lambda$ to itself.
\end{enumerate}
\begin{figure}[H]
\pseudocodeblock{
P(\pw) \< \< P'(\pw') \\ 
\< \sendmessagerightleft*{\text{EKE}} \< \\
K \< \< K' \\
\< \sendmessageright*{h := H_0(K)} \< \\
\text{output }H(K) \< \< \text{output }H(K')
}
\caption{Our counterexample}
\end{figure}
Note that $h$ is a redundant message, in the sense that neither party's output depends on it.
\subsection{The Protocol Is UC-Secure}
\begin{proof}
Since EKE UC-realizes $\Fpake$, we may replace the EKE part in protocol $\Pi$ with $\Fpake$. Call the resulting protocol $\tilde{\Pi}$; we only need to prove that $\tilde{\Pi}$ is UC-secure. In order to distinguish the $\Fpake$ that is a building block of the \emph{real protocol} $\tilde{\Pi}$ and the $\Fpake$ that is the functionality interacting with the simulator in the \emph{ideal world}, we abbreviate the latter as $\func$.

Consider the following simulator $\sim$. Roughly speaking, $\sim$ simulates $\Fpake$ honestly, and samples a random string as $h$, unless $P$'s instance has been successfully attacked --- in which case $\sim$ knows $P$'s EKE key and can simulate $h$ honestly:

\bigskip\noindent\textbf{\underline{Simulator $\sim$}:}
\begin{enumerate}
  \item On $(\NewSession, sid, P, P')$ from $\func$, send the same message from $\Fpake$ to $\adv$. \\
      On $(\NewSession, sid, P', P)$ from $\func$, also send the same message from $\Fpake$ to $\adv$.
  \item On $(\TestPwd, sid, \star, \star)$ from $\adv$ to $\Fpake$, forward this message to $\func$ and sends $\func$'s responde from $\Fpake$ to $\adv$. This defines a state of $P$'s instance ($\fresh$, $\compromised$ or $\interrupted$).
  \item On $(\NewKey, sid, P, K^*)$ from $\adv$ to $\Fpake$,
      \begin{enumerate}[(i)]
        \item If there has been a $(\TestPwd, sid, P, \star)$ command from $\adv$ to $\Fpake$ which resulted in ``correct guess'' (see step 2), then set $h := H_0(K^*)$; otherwise sample $h \gets \{0,1\}^\lambda$. Either way, send $h$ from $P$ to $P'$.
        \item Also compute $SK^* := H(K^*)$ and send $(\NewKey, sid, P, SK^*)$ to $\func$.
      \end{enumerate}
  \item On $(\NewKey, sid, P', (K')^*)$ from $\adv$ to $\Fpake$ and $h^*$ from $\adv$ to $P'$, compute $(SK')^* := H((K')^*)$ and send $(\NewKey, sid, P', (SK')^*)$ to $\func$.
\end{enumerate}

We now show that for any environment $\env$, $\sim$ generates an ideal-world view that is indistinguishable from $\env$'s real-world view. Let $K$ and $K'$ be $P$ and $P'$'s EKE key, respectively, and $SK$ and $SK'$ be $P$ and $P'$'s final output key in $\tilde{\Pi}$, respectively. Define $\mathsf{QueryP}$ be the event that
\begin{itemize}
  \item $\adv$ does not send a $(\TestPwd, sid, P, \star)$ command resulting in ``correct guess'', but
  \item $\adv$ queries $H_0(K)$ or $H(K)$.
\end{itemize}
Define $\mathsf{QueryP'}$ be the event that
\begin{itemize}
  \item $\adv$ does not send a $(\TestPwd, sid, P', \star)$ command resulting in ``correct guess'', but
  \item $\adv$ queries $H(K)$.
\end{itemize}
\begin{claim}
$\Pr[\mathsf{QueryP}]$ and $\Pr[\mathsf{QueryP'}]$ are both negligible.
\end{claim}

If $\adv$ does not send a $(\TestPwd, sid, P, \star)$ command resulting in ``correct guess'', then $\Fpake$ marks $P$'s session as either $\fresh$ or $\interrupted$. Either way, when $K$ is generated it is a uniformly random string in $\{0,1\}^\lambda$ independent of $\env$'s view, and later the only information $\env$ learns about $K$ is $H_0(K)$ (from the protocol message) and $H(K)$ (from $P$'s output key) --- which are independent of $K$ unless and until $\mathsf{QueryP}$ happens. Therefore, $\Pr[\mathsf{QueryP}]$ is upper-bounded by $q_H/2^\lambda$ (where $q_H$ is the total number of $H$ and $H_0$ queries). A similar argument works for $\mathsf{QueryP'}$.
\begin{claim}
If neither $\Pr[\mathsf{QueryP}]$ nor $\Pr[\mathsf{QueryP'}]$ happens, then $\env$'s views in the real world and in the ideal world are identical.
\end{claim}
The argument goes as follows:
\begin{enumerate}
  \item Regarding $\adv$'s interface with $\Fpake$, $\sim$ does nothing other than forwarding $\adv$'s message to $\func$ and forwarding $\func$'s response back to $\adv$. In other words, $\sim$ simulates $\Fpake$ honestly. So $\sim$ generates a view that is identical to the real-world view.
  \item Regarding protocol message $h$, first note that in both worlds this message is generated right after $\adv$ sends $(\NewKey, sid, P, K^*)$ to $\Fpake$.
      \begin{itemize}
        \item If $\adv$ has sent a $(\TestPwd, sid, P, \star)$ command resulting in ``correct guess'', then in the real world $P$'s instance in $\func$ was $\compromised$. This causes $P$'s EKE key $K$ to be equal to $K^*$, so $h = H_0(K) = H_0(K^*)$. This exactly matches how $h$ is generated by $\sim$.
        \item Otherwise, we know that $\adv$ does not query $H_0(K)$ (because $\mathsf{QueryP}$ does not happen). This means that in the real world $h = H_0(K)$ is a uniformly random string in $\{0,1\}^\lambda$ independent of the rest of the game, which again matches how $h$ is generated by $\sim$.
      \end{itemize}
  \item Regarding $P$'s output key $SK$,
      \begin{itemize}
        \item If $\adv$ has sent a $(\TestPwd, sid, P, \star)$ command resulting in ``correct guess'', then as argued above, in the real world $K = K^*$, so $SK = H(K) = H(K^*)$. In the ideal world, $\sim$ computes $SK^* = H(K^*)$, so $SK^* = SK$. Furthermore, $\func$ has marked $P$'s instance $\compromised$ (because $\sim$ has sent a $(\TestPwd, sid, P, \star)$ command resulting in ``correct guess''), so when $\sim$ sends $(\NewKey, sid, P, SK^*)$ to $\func$, $\func$ outputs $SK^*$ to $P$. In sum, in both worlds $P$'s output key is $SK = H(K^*)$, so there is no difference.
        \item Otherwise, we know that $\adv$ does not query $H(K)$ (because $\mathsf{QueryP}$ does not happen). This means that in the real world $SK$ is a uniformly random string in $\{0,1\}^\lambda$ independent of the rest of the game. In the ideal world, $\func$ has marked $P$'s instance $\fresh$ or $\interrupted$, so when $\sim$ sends $(\NewKey, sid, P, SK^*)$ to $\func$, $\func$ ignores $SK^*$ and freshly samples a uniformly random string in $\{0,1\}^\lambda$ as $P$'s output key. In sum, in both worlds $P$'s output key is $SK \gets \{0,1\}^\lambda$, so there is no difference.
      \end{itemize}
      The same argument can be made for $P'$'s output key $SK'$.
\end{enumerate}

Combining the two claims above immediately yields the UC-security of $\tilde{\Pi}$.
\end{proof}
\subsection{The Protocol Is Not KOY-Secure}
\begin{proof}
Consider the following adversary $\adv$. Roughly speaking, $\adv$ assumes the role of $P'$ and runs $P'$'s algorithm on all candidate passwords, and uses $h$ to find out which one is the correct password (after which the attack becomes trivial):

\bigskip\noindent\textbf{\underline{Adversary $\adv$}:}

\medskip
For all $\pw_i \in D$, $\adv$ performs the following steps until it finds $\pw^*$:
\begin{enumerate}
  \item Initiate $P$'s instance $P^i$ by sending $\Send(P^i, \text{start})$ to the game challenger.
  \item On $c$ from the game challenger, sample $y \gets \Z_p$ and
      \begin{enumerate}[(i)]
        \item Compute $Y^* := \ICEnc(\pw_i, g^y)$ and send $\Send((P')^i, P^i, Y^*)$ to the game challenger.
        \item Also, compute $X := \ICDec(\pw_i, c)$ and then $K^* := X^y$.
      \end{enumerate}
  \item On $h$ from the game challenger, check if $h = H_0(K^*)$. If so, set $\pw^* := \pw_i$ and break the for-loop. Otherwise move on to the next element in $D$.
\end{enumerate}

See below for a graphic illustration of steps 1--3 of $\adv$.
\begin{figure}[H]
\pseudocodeblock{
P(\pw) \< \< \adv(\pw_i) \\ 
\< \sendmessagerightleft*{\text{EKE}} \< \\
K \< \< K^* \\
\< \sendmessageright*{h := H_0(K)} \< \\
\< \< h \stackrel{?}= H_0(K^*) \\
\< \< \text{if so, }\pw^* := \pw_i;\text{ break}
}
\end{figure}

After the for-loop ends, $\adv$ finds the $i$ such that $\pw^* = \pw_i$ and performs the following step:
\begin{enumerate}
\setcounter{enumi}{3}
  \item Send $\Test(P^i)$ to the game challenger. If the result is equal to $H(K^*)$, output 1; otherwise output 0.
\end{enumerate}

Note that {\color{blue}$\adv$ only sends a single $\Test$ command, so it only does 1 active session attack and its ``baseline'' winning probability is $1/2 + 1/2|D|$}. However, we show that $\adv$'s winning probability is in fact overwhelming:
\begin{claim}
The probability that $\pw^* = \pw$, i.e., $\adv$ finds the correct password, is overwhelming.
\end{claim}
Recall EKE has the property that the two parties' output keys are equal \emph{if and only if} their passwords match. Note that $\adv$'s steps 1--2 are exactly what an honest $P'$ would to on password $\pw_i$; therefore, if $\pw_i = \pw$ then $P$'s EKE key $K \neq K^*$, so $h = H_0(K)$ is not equal to $H_0(K^*)$ except with probability $1/2^\lambda$, causing $\adv$ to move on to the next candidate password. If $\pw_i = \pw$ then $P$'s EKE key $K = K^*$, so $h = H_0(K) = H_0(K')$, causing $\adv$ to break the for-loop and set $\pw^* := \pw_i = \pw$. Overall, $\adv$ sets $\pw^* = \pw$ except with probability up to $(|D|-1)/2^\lambda$.
\begin{claim}
If $\pw^* = \pw$, then $\adv$ wins with overwhelming probability.
\end{claim}
Suppose $b = 1$. Then when $\adv$ sends $\Test(P^i)$, the game challenger returns $P^i$'s output key $H(K)$. But if $\pw^* = \pw$ then $K^* = K$, so $H(K) = H(K^*)$ and $\adv$ outputs 1.

Now suppose $b = 0$. Then when $\adv$ sends $\Test(P^i)$, the game challenger returns a uniformly random string in $\{0,1\}^\lambda$ independent of the rest of the game. Therefore, $\adv$ outputs 1 with probability $1/2^\lambda$.

In sum, $\adv$ wins with probability 1 when $b = 1$ and $1-1/2^\lambda$ when $b = 0$. Its overall winning probability is $1-1/2^{\lambda+1}$.

\medskip
Combining the two claim above immediately yields that $\adv$ wins with overwhelming probability, so $\Pi$ is not KOY-secure.\footnote{Here we need $|D| \geq 2$; otherwise $\adv$'s ``baseline'' winning probability is 1, so winning with overwhelming probability does not violate the security definition.}
\end{proof}
\paragraph{$\Pi$ is still BPR-secure.}
Our attack above does not violate the BPR-security of $\Pi$. This is because the sentence in blue does not hold anymore: $\adv$ sends up to $|D|$ $\Send$ commands, so in BPR the number of online attacks is up to $|D|$ and thus $\adv$'s ``baseline'' winning probability is $1/2 + |D|/2|D| = 1$.

\section{Implicit-Only UC-Security Implies KOY-Security}
\label{sec:implicit-only}
Our main result in this section is:
\begin{theorem}
\label{thm:implicit-only}
Suppose $\Pi$ is a PAKE protocol that UC-realizes $\Fipake$. Then $\Pi$ is KOY-secure.
\end{theorem}
We need the following technical lemma from \cite[Lemma~A.1]{EC:CHKLM05} (for a detailed proof, see \cite[Appendix~A]{PKC:RoyXu23}):
\begin{lemma}
\label{lem:spuriousguess}
Suppose $\Pi$ is a PAKE protocol that UC-realizes $\Fipake$. Fix any ``successful'' simulator $\sim$ that, for any PPT environment, generates a view indistinguishable from the environment's real-world view in $\Pi$; and any PPT environment $\env$. Let $\SpuriousGuess$ be the event that there is a session $sid$ in which $\env$ instructs the adversary to be passive, yet $\sim$ sends $(\TestPwd, sid, \star, \star)$ to $\Fipake$. Then $\Pr[\SpuriousGuess] \leq \negl$.\footnote{Technically speaking $\Pi$ must be correct, i.e., if there is no adversary then the two protocol parties output the same key. See \cite{PKC:RoyXu23} for a discussion. In this work we always assume PAKE protocols to be correct.}
\end{lemma}
We are now ready to prove \Cref{thm:implicit-only}.
\begin{proof}
Consider any PPT adversary $\widehat{\adv}$ in the KOY-security game for $\Pi$, and let $q$ be the number of online attacks by $\widehat{\adv}$; we want to upper-bound $\widehat{\adv}$'s winning probability. (We use $\widehat{\adv}$ instead of $\adv$ to emphasize that it is the adversary in the game-based definition, not the UC adversary.) The outline of the proof is as follows:
\begin{enumerate}
  \item We first ``translate'' $\widehat{\adv}$ into a UC environment $\env$ that essentially does what $\widehat{\adv}$ does during protocol execution. $\env$ outputs 1 if and only if $\widehat{\adv}$ wins.
  \item Since $\env$ simulates the KOY-security game for $\widehat{\adv}$, the probability that $\env$ outputs 1 in the real world is equal to $\widehat{\adv}$'s winning probability.
  \item However, in the ideal world $\widehat{\adv}$'s winning probability is limited by how much it learns about the password. But $\widehat{\adv}$ gains some information about the password only if \emph{both} (1) the simulator does a successful online attack, and (2) $\widehat{\adv}$ sees the attacked instance's output key. The number of occurrences of (1) and (2) is exactly $q$.
  \item Therefore, $\widehat{\adv}$'s winning probability must be close to the ``baseline''.
\end{enumerate}
Steps 1 and 2 are essentially identical to the proof of UC-security $\Rightarrow$ BPR-security \cite{EC:CHKLM05}; the main technical challenge lies in step 3.
\paragraph{``Translating'' $\widehat{\adv}$ into a UC environment.}
Consider the following UC environment $\env$:

\bigskip\noindent\textbf{\underline{Environment $\env$}:}
\begin{enumerate}
\setcounter{enumi}{-1}
  \item Sample $b \gets \{0,1\}$ and invoke $\widehat{\adv}$.
  \item For each pair of instances $(P^i, (P')^j)$, sample a password $\pw_{i,j} = \pw_{j,i} \gets D$.
  \item On $\Execute(P^i, (P')^j)$ from $\widehat{\adv}$, pick a fresh $sid_{i,j} = sid_{j,i}$ and send $(\NewSession, \allowbreak sid_{i,j}, P', \pw_{i,j})$ to $P'$ and $(\NewSession, sid_{i,j}, P, \pw_{i,j})$ to $P$. (Below we may simply write $sid$ and $\pw$ when $i,j$ are clear from the context.) Then instruct the UC PAKE adversary $\adv$ to be passive in this session. After the session completes, provide $\widehat{\adv}$ with its transcript. (Note that $P^i$ is an instance in the KOY-security game simulated by $\env$, whereas $P$ is a UC protocol party that interacts with $\env$.)
  \item On $\Send(P^i, (P')^j, \text{start})$ from $\widehat{\adv}$, if there has been a $\Send((P')^j, P^i, \text{start})$ command, then $sid_{j,i} = sid_{i,j}$ must have been defined. In this case, send $(\NewSession, sid_{i,j}, P', \pw)$ to $P$. Otherwise pick a fresh $sid_{i,j} = sid_{j,i}$ and send $(\NewSession, sid_{i,j}, P', \pw)$ to $P$. \\
      On $\Send(P^i, (P')^j, m^*)$ from $\widehat{\adv}$, instruct the UC PAKE adversary $\adv$ to send a message $(sid, m^*)$ to $P'$. \\
      In all cases above, forward the message $\adv$ receives to $\widehat{\adv}$.
  \item On $\Reveal(P^i)$ from $\widehat{\adv}$, fetch $sid_{i,j}$ (where $j$ is the index of instance $(P')^j$ that $P^i$ communicates with). If $P$ has output $(sid_{i,j}, SK)$ to $\env$, return $SK$; otherwise return $\bot$.
  \item $\Test(P^i)$ from $\widehat{\adv}$ is handled in the same way as $\Reveal(P^i)$, except that ``return $SK$'' is replaced by ``return $SK$ if $b = 1$ and a random string in $\{0,1\}^\lambda$ if $b = 0$''.
  \item When $\widehat{\adv}$ outputs a bit $b^*$, output 1 if $b^* = b$ and 0 otherwise.
\end{enumerate}
\paragraph{Analysis in the real world.}
If $\env$ is in the real world, then it exactly simulates $\Pi$'s KOY-security game for $\widehat{\adv}$. We have 
\[
\Pr[\env\text{ outputs }1\text{ in the real world}] = \Pr[b^* = b\text{ in the real world}] = \Pr[\widehat{\adv}\text{ wins}].
\]
\paragraph{Analysis in the ideal world.}
Now suppose $\env$ is in the ideal world. Let $\sim$ be the simulator that generates a ideal-world view indistinguishable from $\env$'s real-world view. We now upper-bound $\Pr[b^* = b\text{ in the ideal world}]$.

Define ``bad event''
\begin{align*}
\CorrectGuess : \sim\text{ sends }(\TestPwd, sid_{i,j}, P, \pw)\text{ to }\Fipake \\
\text{where }\widehat{\adv}\text{ performs an online attack on }P^i,
\end{align*}
i.e., $\CorrectGuess$ happens if $\sim$ tests $\pw$ on an instance on which $\widehat{\adv}$ performs an online attack. (Intuitively, $\CorrectGuess$ happens if $\widehat{\adv}$'s protocol messages contain password guess $\pw$, which is extracted by $\sim$; and eventually $\widehat{\adv}$ learns this instance's output key.)
\begin{claim}
\label{clm:tested}
Suppose $\SpuriousGuess$ does not happen. Then $\Pr[\CorrectGuess] \leq \cfrac{q}{|D|}$.
\end{claim}
This claim is the crux of the entire proof (and where we crucially deviate from \cite{EC:CHKLM05}). The key point is to \emph{view $\widehat{\adv}$ and $\sim$ combined as a single entity, and see what information they receive about $\pw$}. In fact there is only a single cycle through which $\widehat{\adv}$ and $\sim$ might (jointly) learn some information about $\pw$:
\begin{enumerate}
  \item For a certain instance, $\sim$ submits a password guess $\pw^*$ (via $\TestPwd$) and a key $K^*$ (via $\NewKey$) to $\Fipake$.
  \item $\Fipake$ generates key $K$ which is equal to $K^*$ if $\pw^* = \pw$ and uniformly random otherwise. $\Fipake$ then sends $K$ to the UC protocol party $P$.
  \item $P$ forwards $K$ to $\env$.
  \item When $\widehat{\adv}$ sends a $\Reveal$ command for this instance, or a $\Test$ command in the case of $b = 1$, $\env$ returns $K$ to $\widehat{\adv}$.
\end{enumerate}
(Other information $\widehat{\adv}$ and $\sim$ may receive includes $\NewSession$ commands from $\Fipake$ to $\sim$, indicating that a session has begun. This message is of course independent of $\pw$. Note in particular that protocol messages $\widehat{\adv}$ receives on $\Execute$ and $\Send$ commands are in fact simulated by $\sim$: upon receiving such a command, $\env$ lets $\sim$ simulate protocol messages for the UC PAKE adversary $\adv$, which $\env$ forwards to $\widehat{\adv}$. Therefore, {\color{blue}when we view $\widehat{\adv}$ and $\sim$ jointly, $\Execute$ and $\Send$ commands (without $\Reveal$ or $\Test$) provide no information about $\pw$}.)

Let's call steps 1--4 above a ``cycle''. In short, the game becomes: in each cycle $\{\widehat{\adv}, \sim\}$ may specify $\pw^* \in D$ and $K^*$, and receive $K^*$ if $\pw^* = \pw$ and a uniformly random $K$ otherwise. $\CorrectGuess$ is triggered when $\{\widehat{\adv}, \sim\}$ specifies $\pw$. In each cycle the only information $\{\widehat{\adv}, \sim\}$ learns about $\pw$ is whether $\pw^* = \pw$, so $\Pr[\CorrectGuess]$ is upper-bounded by $\text{(number of cycles)}/|D|$.

How many cycles can $\{\widehat{\adv}, \sim\}$ complete? First, $\widehat{\adv}$ must send a $\Reveal$ or a $\Test$ command on this instance (otherwise $\{\widehat{\adv}, \sim\}$ does not learn $K$ and thus learns no information about $\pw$). Second, if $\SpuriousGuess$ does not happen then this instance must be a $\Send$ instance (otherwise $\sim$ does not even send a $\TestPwd$ command, i.e., $\{\widehat{\adv}, \sim\}$ does not even specify $\pw^*$). But a cycle that satisfies these conditions is exactly an online attack. Therefore, the number of cycles is at most $q$.

We conclude that
\[
\Pr[\CorrectGuess] \leq \frac{\text{number of cycles}}{|D|} \leq \frac{q}{|D|}.
\]
\begin{claim}
Suppose again $\SpuriousGuess$ does not happen. If $\CorrectGuess$ also does not happen, then $b^* = b$ with probability $1/2$.
\end{claim}
Consider the instance for which $\widehat{\adv}$ sends $\Test$. If it is an $\Execute$ instance, then $\sim$ does not send $\TestPwd$ (because $\SpuriousGuess$ does not happen), so this session is $\fresh$ and outputs a uniformly random key. If it is a $\Send$ instance, then this instance is actively attacked, so $\sim$ does not send $\TestPwd$ on the correct password (because $\CorrectGuess$ does not happen), so this session is $\fresh$ or $\interrupted$ and outputs a uniformly random key. This means that $b$ is independent of $\widehat{\adv}$'s view, so $\Pr[b^* = b] = 1/2$.

\medskip
Combining the two claims above, we get
\begin{align*}
&\Pr[\env\text{ outputs }1\text{ in the real world}] \\
\leq& \Pr[\CorrectGuess \mid \overline{\SpuriousGuess}] + \Pr[b^* = b \land \overline{\CorrectGuess} \mid \overline{\SpuriousGuess}] + \Pr[\SpuriousGuess] \\
\leq& \Pr[\CorrectGuess \mid \overline{\SpuriousGuess}] + \frac{1}{2}(1 - \Pr[\CorrectGuess \mid \overline{\SpuriousGuess}]) + \Pr[\SpuriousGuess] \\
=& \frac{1}{2} + \frac{1}{2}\Pr[\CorrectGuess \mid \overline{\SpuriousGuess}] + \Pr[\SpuriousGuess] \\
\leq& \frac{1}{2} + \frac{q}{2|D|} + \Pr[\SpuriousGuess] \\
\leq& \frac{1}{2} + \frac{q}{2|D|} + \negl,
\end{align*}
where the last inequality is due to \Cref{lem:spuriousguess}.
\paragraph{Putting it together.}
In sum, $\env$'s distinguishing advantage is at least
\[
\Pr[\widehat{\adv}\text{ wins}] - \frac{1}{2} - \frac{q}{2|D|} - \negl,
\]
which is negligible since $\Pi$ UC-realizes $\Fipake$. So
\[
\Pr[\widehat{\adv}\text{ wins}] \geq \frac{1}{2} + \frac{q}{2|D|} + \negl,
\]
completing the proof.
\end{proof}
\paragraph{Recovering the UC-security $\Rightarrow$ BPR-security proof.} One might ask where the proof breaks down if $\Pi$ only realizes $\Fpake$. The reason is that the statement in blue does not hold anymore, as there is another cycle that allows the combination of $\widehat{\adv}$ and $\sim$ to learn some information about $\pw$:
\begin{enumerate}
  \item For a certain instance, $\sim$ submits a password guess $\pw^*$ (via $\TestPwd$) to $\Fpake$.
  \item $\Fpake$ returns ``correct/wrong guess'' to $\sim$.
\end{enumerate}
Crucially, this cycle does not require $\widehat{\adv}$ to send $\Reveal$ or $\Test$, i.e., $\sim$ can ``help'' $\widehat{\adv}$ gain some information about $\pw$ without $\widehat{\adv}$ performing an online attack (per the KOY notion). This is exactly what our counterexample in \Cref{sec:counterexample} exploits.

On the other hand, if we take the BPR notion (where a $\Send$ command already counts as an online attack), then the cycle above still requires an online attack (assuming $\SpuriousGuess$ does not happen) and the rest of the proof still goes through. In this way we recover the security proof of UC-security $\Rightarrow$ BPR-security \cite{EC:CHKLM05}.

\section{Discussion}
\label{sec:conclusion}
How to interpret our results? Let's first look at the difference between the two game-based security definitions. In BPR-security, any $\Send$ command to an instance counts as an online attack; whereas in KOY-security, an online attack must also $\Reveal$ or $\Test$ the instance's key. In other words, BPR-security allows the protocol transcript alone to leak some information about the password in an active attack, whereas in KOY-security only the protocol transcript {\color{blue}plus the attacked instance's output key} may tell the adversary something about the password --- which is inevitable.\footnote{Note that even in BPR-security, the transcript should not leak any information about the password if the adversary is passive (i.e., the adversary sends an $\Execute$ command).} Our counterexample in \Cref{sec:counterexample} exactly exploits this: the protocol message $h$ allows the adversary to test password guesses --- and eventually find the correct password --- by sending $\Send$ commands only, violating KOY- but not BPR-security.

The difference between UC- and implicit-only UC-security can be interpreted in a similar way. Consider the two ideal functionalities: $\Fpake$ allows the ideal adversary to check if its password guess is correct or not via a $\TestPwd$ command, which means that in the real world the adversary can learn this bit of information by observing the transcript in an active attack; whereas in $\Fipake$ the adversary does not learn this bit. In fact, in $\Fipake$ \emph{the ideal adversary never learns any information about any instance's password}, which is crucial in our proof in \Cref{sec:implicit-only}.

Our conclusion is that KOY- and implicit-only UC-security notions strengthen their standard counterparts in the same manner, namely
\begin{framed}
\begin{displayquote}
  \emph{KOY- and implicit-only UC-security disallow the adversary from learning whether an online attack has succeeded or not by merely examining the protocol transcript (without seeing the attacked instance's output key).}
\end{displayquote}
\end{framed}

Finally, we explain why this property is called ``implicit-only''. PAKE with explicit authentication allows each protocol party to check whether the authentication has been successful, i.e., whether its output key is equal to its counterparty's. But if an honest party can learn this, then an adversary that impersonates this party on a password guess can also learn this; that is, an adversary that communicates with Alice by running Bob's algorithm on password guess $\pw^*$, can learn whether it computes the same key as Alice's output key --- which in turn is equivalent to $\pw^*$ being equal to Alice's password. In this way, the adversary can test a password guess without seeing Alice's output key, violating KOY- or implicit-only UC-security. In other words, \emph{a PAKE protocol with explicit authentication cannot be implicit-only UC-secure} --- indeed, the message $h$ in our counterexample is exactly what a protocol party would send if we wanted to add explicit authentication to EKE.

\bibliographystyle{alpha}
\bibliography{bib.bib}
\end{document} 